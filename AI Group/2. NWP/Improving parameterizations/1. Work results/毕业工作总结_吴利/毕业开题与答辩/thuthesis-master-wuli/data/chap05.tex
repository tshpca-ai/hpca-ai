\chapter{总结与展望}
\label{cha:intro}

\section{总结}
近年来由于极端气候事件频发,引发了人们对气候预报的关注。气候系统模式作为模拟气候的数值模型,是预测未来气候的强有力工具。然而由于气候系统模式作为真实气候的数值模拟存在着诸多的不确定性。集合预报是降低模拟不确定性的有效方法,从20世纪90年代以来集合预报方法逐渐应用到世界各地的天气预报业务系统中。由于气候系统模式运行一次需要极高的计算代价,气候集合方法的研究相对天气集合方法有一定的滞后性。但是随着高性能计算机的高速发展,气候预报也逐渐由过去的单一的确定性预报走向了集合预报。目前的气候集合预报还存在着以下两个重要的问题,一是气候系统模式中参数不确定性问题严峻,目前的集合方法未能彻底解决这一问题。二是气候集合初值扰动方法研究仍然处在起步阶段。针对这两个问题,本文提出了一种气候系统模式参数优化与初值集合方法相结合的集合技术。在参数优化方面本文根据气候系统模式参数维度高,非线性强,运行一次需要极高代价的特点提出了高效的代理模式参数优化思路。并针对各种不同的类型参数优化问题,设计了单目标优化、多目标优化和有约束优化方法。然后将其在复杂数学函数和单柱大气模式中进行了验证。验证结果表明新提出的方法在精度和收敛性方面都表现良好。在气候集合初值扰动方法上本文设计了基于BGM的初值扰动方案,并将此方案与国家气候中心所使用的LAF方法做了详细的对比分析,发现本文设计的BGM相对LAF方法能够进一步提升集合预报的能力。另外本文提出了将参数优化和初值扰动技术相结合的BGMOPT集合方法,结果表明这一新的方法对气候系统模式模拟精度有着显著的提升。最后本文设计了自动化的集合预测系统,并对当前确定性集合集成方法进行了改进,将其从简单的集合平均改进为基于机器学习的多成员集合结果修正方法。本文的主要贡献有以下几个方面:

(1)详细分析了气候系统模式中存在的各类不确定性参数优化问题,并针对性地提出了高效的基于多层感知机神经网络的代理模式优化方法。此优化方法单步更新优化策略相对于常用的种群更新策略的进化算法来说收敛性更快。而相对于基于传统统计回归模型的代理模式优化而言又提升了优化策略的决策能力。在TWP-ICE单柱大气模式中本文的多目标优化方法在优化精度更高的情况下相对于NSGAIII方法收敛速度可提升5倍以上。

(2)设计了一种气候集合初值扰动方法,该扰动方法在2000至2014年5月到8月的回报试验中预测能力超过了国家气候中心气候预测系统所使用的LAF方法。预报起第一个月内环流场改进明显。500hpa位势高度相对于LAF方法在预报开始的第一个月内改进了10.2\%。200hp纬向风在预报起的第一个月内改进6.4\%,850hpa纬向风在预报的第一个月内改进7\%,其中500hpa位势高度和850hpa纬向风的改进可以持续到整个夏季。在对我国气候影响较大的亚洲和西太平洋区域,BGM集合方法预测的5月的降水和地表温度都优于LAF方法。

(3)提出了结合参数优化和初值扰动的集合预测方法(BGMOPT)并将其应用在BCC-CSM气候系统模式的降水预测中。本文先针对六年长期的模拟对不确定参数进行优化,然后将优化的参数用于初值集合预报。在以辐射平衡为约束,MJO和EASM模拟能力为目标的有约束多目标优化问题中,对BCC-CSM大气模块中的深浅对流,云等物理参数化方案中的不确定参数进行了优化,优化后的模式在满足长期模拟大气顶辐射平衡的基础上,MJO和EASM的模拟能力相对于默认实验提高明显。改进参数后的初值集合方法(BGMOPT)相对于原LAF方法在2008年12月到2009年3月的回报案例中的平均降水能力提升15\%。

(4) 提出了结合观测和模式输出数据特征的机器学习集合预测修正方法,相对于集合平均方法可将ENSO预测能力提升32\%。

(5)集合预测系统的设计与实现。集合预测系统包括参数优化模块、集合扰动生成模块、集合预测模块、集合集成与后处理模块、数据库模块,将繁重的参数优化和集合扰动生成及预测过程自动化。

%新的方法的集成结果相对于原方法预测能力提升**倍。

\section{下一步工作}
本文针对气候系统模式预测提出了基于参数优化的初值集合预测方法,在本文工作的基础上可以进一步进行的研究有以下几个方面:

(1) 时空变化的参数优化。目前的气候系统模式的参数优化方法都是利用长期模拟对不确定参数进行优化。然而在实际的预测中,不同的时间和不同地点最优的不确定参数不一定完全一致。这个时候就需要时空变化的参数优化方法使得某个时间某个地点的气候预测更加准确。
 
(2) 多模式集合的气候预测。本文的气候预测集合方案研究都是在BCC-CSM气候系统模式的基础上。然而国际上多模式集合的研究也正在发展当中。由于气候预测问题存在着巨大的挑战,单一的气候系统模式在气候预测方面总是有相对欠缺的地方,对于模式层面的不确定性表征存在或多或少的不足。而多模式集合方法利用多个优秀的气候系统模式共同来生成集合,集合成员的分散性会更好,表征能力更强,可以取得更好的气候预测效果。






