\begin{resume}

  \resumeitem{个人简历}

  1992 年 10 月 05 日出生于 安徽 省 枞阳 县。

  2011 年 9 月考入青海大学计算机科学与技术系计算机技术与应用专业,2015年 7 月本科毕业并获得工学学士学位。

  2016 年 9 月免试进入 清华大学 计算机科学与技术 系攻读 硕士 学位至今。

  \researchitem{发表的学术论文} % 发表的和录用的合在一起

  % 1. 已经刊载的学术论文(本人是第一作者,或者导师为第一作者本人是第二作者)
  \begin{publications}
    \item \textbf{吴利},黄欣,薛巍. 基于多层感知机代理模式的地球系统模式物理参数优化方法[J].电子技术应用.
    \item Xinliang Wang, Weifeng Liu, Wei Xue, and \textbf{Li Wu}.swSpTRSV: a fast sparse triangular solve with sparse level tile layout on sunway architectures[C].Principles and Practice of Parallel Programming. ACM, 2018:338-353.
  \end{publications}
  
  \researchitem{在审的论文} % 发表的和录用的合在一起

  % 1. 已经刊载的学术论文(本人是第一作者,或者导师为第一作者本人是第二作者)
  \begin{publications}
    \item \textbf{Li Wu}, Tao Zhang, Yi Qin, Wei Xue. An effective parameter optimization with radiation balance constraint in CAM5[J].Geoscientific Model Development Discussion.
    \item \textbf{Li Wu}, Xiaoge Xin, Wei Xue, Yanjie Cheng, Xiangwen Liu,and Tongwen Wu.Application of breeding method in Beijing Climate Center Climate System Model for ensemble subseasonal forecast[J].Advances in Atmospheric Sciences.
  \end{publications}

  \researchitem{口头报告与海报} % 发表的和录用的合在一起

  % 1. 已经刊载的学术论文(本人是第一作者,或者导师为第一作者本人是第二作者)
  \begin{publications}
    %\item \textbf{Li Wu} ,Tao Zhang, Yi Qin, Yanluan Lin, Wei Xue,Minghua Zhang. An effective parameter optimization with radiation balance constraint in CAM5. AGU Fall Meeting 2017(Poster).
    \item \textbf{Li Wu} ,Xiaoge Xin, Yanjie Cheng, Wei Xue. Application of Breeding of Growing Mode ensemble method in seasonal prediction of climate system model. AGU Fall Meeting 2018(Poster).
  \end{publications}
  
  \researchitem{参与项目情况} % 发表的和录用的合在一起
  % 1. 已经刊载的学术论文(本人是第一作者,或者导师为第一作者本人是第二作者)
  \begin{publications}
    \item 2017年7月至今:国家重点研发计划"地球系统模式公共软件平台研发"(编号:2017YFA0604500) 
    \item 2016年9月至今:国家重点研发计划课题"基于高分辨率气候系统模式的无缝隙气候预测系统研制与评估"(编号:2016YFA0602103)
  \end{publications}
  

\end{resume}
